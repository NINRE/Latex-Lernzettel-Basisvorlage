\documentclass[a4paper,12pt]{article}

% --- Pakete ---
\usepackage[utf8]{inputenc}
\usepackage[T1]{fontenc}
\usepackage[ngerman]{babel}
\usepackage[a4paper,margin=2cm]{geometry}
\usepackage{amsmath,amssymb,mathtools}
\usepackage{tcolorbox}
\usepackage{xcolor}
\usepackage{hyperref}
\usepackage{bookmark}
\usepackage{enumitem}
\usepackage{geometry}
\usepackage{fancyhdr}
\usepackage{graphicx}
\usepackage{multicol}

\pagestyle{fancy}
\fancyhf{}
\rhead{ Lernzettel}
\lhead{}
\rfoot{Seite \thepage}
\setlength{\headheight}{14.49998pt}
\addtolength{\topmargin}{-2.49998pt}

% --- Farben & Design ---
\definecolor{myblue}{RGB}{0,102,204}
\hypersetup{
    colorlinks=true,
    linkcolor=myblue,
    urlcolor=myblue
}

% --- Box für Definitionen ---
\newtcolorbox{definitionbox}{
  colback=blue!5!white,
  colframe=blue!75!black,
  title=Definition
}

% --- Absatzformatierung ---
\setlength{\parskip}{0.3em}
\setlength{\parindent}{0pt}

% --- Dokumentinformationen ---
\title{Lernzettel -- [Fach oder Thema]}
\author{Name des Autors}
\date{}

\begin{document}

\maketitle
\tableofcontents
\newpage

\section{Thema 1: ...}

\begin{definitionbox}
...
\end{definitionbox}

Wichtige Begriffe:
\begin{itemize}[leftmargin=*]
  \item 1
  \item 2
  \item 3
\end{itemize}



\section{Thema 2: ...}

\begin{definitionbox}
...
\end{definitionbox}


\begin{itemize}[leftmargin=*]
  \item 1
  \item 2
  \item 3
\end{itemize}

\section{Thema 3: ...}

\begin{definitionbox}
  ...
\end{definitionbox}  
\begin{itemize}[leftmargin=*]
  \item 1
  \item 2
  \item 3
\end{itemize}


\begin{tcolorbox}[title=Beispiel, colback=blue!5!white, colframe=blue!75!black]
  ...
\end{tcolorbox}
  
\begin{tcolorbox}[title=Beispiel, colframe=red!75!black] 
     \begin{itemize}
    \item 1
    \item 2
    \item 3
    \item 4
     \end{itemize}      
\end{tcolorbox}
\end{document}